\section{Conclusiones}

% relación con la intro y problemas a intentar resolver
En este trabajo se planteó llevar a cabo una investigación en la que mediante la aplicación de técnicas de Aprendizaje Profundo se realice la tarea de clasificar escenas relacionadas a propiedades inmuebles.

% data recolectada (mencionar variabilidad entre datasets)
A partir de la revisión de antecedentes se obtuvieron dos conjuntos de datos con los que se plantearon diferentes experimentos, por un lado el realizado en \cite{vision_based_real_estate_price_estimation} y por otro el de \cite{lstm_real_estate}. Ambos con diferentes cantidades de imágenes pero también con diferentes calidades en relación a las imágenes que contienen. Por un lado, el banco de datos del trabajo \cite{lstm_real_estate} tiene menor cantidad de imágenes, representa un 5\% de las del trabajo \cite{vision_based_real_estate_price_estimation}, aunque de igual modo sus imágenes resultan más representativas a cada escena que componen (ver \ref{anexo4:random_img_por_dataset}). Mencionado este punto, los resultados de cada experimento estuvieron sujetos a la calidad de estos bancos de datos de escenas, en el experimento \ref{sssec:exp1} que se utilizó el conjunto de datos de \cite{lstm_real_estate} se alcanzaron mejores resultados que en el mejor de los casos en los que se usó \cite{vision_based_real_estate_price_estimation}; estos bancos de datos se eligieron a sabiendas de la calidad de cada uno debido a que en ambientes productivos pueden ocurrir ambos casos: que las imágenes sean más representativas a su escena o que tengan cierto "ruido", como ser fotos sacadas de lejos, desde ángulos poco vistozos, escenas que no contengan mucha información, escenas de un mismo tipo pero que por pertenecer a diferentes culturas no contienen las mismas entidades, etcétera. 

% logros y aportes esenciales (investigación + modelos) 
En este proyecto se realizaron y validaron hipótesis que lleven a poder resolver el problema inicial planteado: la clasificación de escenas de propiedades inmuebles mediante técnicas de aprendizaje profundo.
Los resultados de los experimentos permiten dar a luz conocimiento sobre algunos posibles enfoques a la hora de enfrentar este problema como son: entrenar una red propia como también tomar provecho de redes preentrenadas. 

A partir de lo demostrado en el análisis de la sección \ref{error_analysis} es posible decir que:
\begin{itemize}
	\item Los modelos no aciertan con el mismo ratio en todas las categorías.
	\item Aunque existan errores en el modelo, la distribución de las probabilidades con las que se predicen los casos mal etiquetados muestra que los modelos no tienen grandes certezas de lo que están prediciendo.
	\item Para ambientes productivos se podría setear un \(umbral\) con el cual se acepten las predicciones realizadas o no, de manera que si no se aceptan las predicciones se pueda tomar acción ante cada imagen.
	\item Ante casos en los que se espera que los modelos funcionen correctamente 
	queda demostrado que la exactitud categórica mejora de manera significativa, por lo que se asume que una de las posibles razones de que un modelo no alcance resultados correctos es la calidad de la escena que se busca clasificar.
\end{itemize} 

Por otro lado, disponibilizar modelos listos para realizar predicciones sobre las clases elegidas en \ref{ssec:limitaciones} resulta uno de los grandes logros de este trabajo ya que pueden servir tanto para utilizarlos en ambientes productivos como para ser el puntapié inicial a nuevas investigaciones, proyectos, modelos e incluso la aplicación de Aprendizaje Mediante Transferencia a partir de los mismos.

% lineamientos futuros >> más datos, otras arquitecturas, más etiquetas sobre los mismos datos
Como quedó demostrado en \ref{sec:revision_antecedentes} el espectro de posibles enfoques para solucionar este problema incluye obviamente variantes de Redes Neuronales Convolucionales, pero también otros tipos de redes pudiesen ser capaces de dar mejor respuesta a la clasificación de escenas como pueden ser las Redes Neuronales Recurrentes o las planteadas (pero aún no incluidas en librerías estándar) Redes Neuronales Convolucionales Aleatorias Potenciadas por el Gradiente; con esto se quiere decir que es posible que otras arquitecturas (ya existentes o no) la tarea se pueda realizar con mayor exactitud. 


Algunos trabajos a futuro que se podrían plantear son: 
\begin{itemize}
	\item Investigaciones sobre el error que cometen los modelos planteados en este trabajo, desde más alto a más bajo nivel, es decir: desde las categorías que se predicen con menor exactitud entre el resto hasta las activaciones de las neuronas en las capas convolucionales, con el fin de intentar solucionar estos problemas.
	\item Recolección de datos relacionada a una cultura o a un país en	específico para entrenar modelos diferenciados
	\item En el experimento \ref{sssec:exp5} se realizó Aprendizaje Mediante Transferencia utilizando la red ImageNetCNN que fue entrenada con cientos de objetos que no necesariamente pertenecen al contexto de las escenas elegidas en \ref{ssec:limitaciones}, por lo que tendría sentido reentrenar esta red con objetos que sean apacibles de aparecer en las escenas seleccionadas y luego a partir de esa nueva red dar lugar al Aprendizaje por Transferencia.
	\item Investigaciones haciendo Aprendizaje por Transferencia utilizando versiones de redes preentrenadas con otras arquitecturas base como pueden ser ResNet o InceptionV3, tanto para PlacesCNN como para ImageNetCNN.
	\item En el experimento \ref{sssec:exp6} se observó una mejora mínima en los resultados en relación al modelo del experimento \ref{sssec:exp4}, siendo que la única diferencia entre ambos es que al conjunto de datos se le aplicó un filtro para mejorar la luminosidad y contraste de las entidades presentes. Esta situación es un posible indicador que la aplicación de otros filtros o realizar Aumentación de Datos podría llevar a mejores resultados.
	\item Es cierto que los experimentos \ref{sssec:exp4}, \ref{sssec:exp5} y \ref{sssec:exp6} obtuvieron resultados muy similares por lo que podría plantearse revisar el conjunto de datos y realizar una limpieza de aquellas imágenes que contengan demasiado ruido o que sean menos representativas de cada escena bajo algún criterio (calidad de la imagen, cantidad de objetos pertenecientes a la escena encontrados, etc) ya que estos casos podrían estar impidiendo el correcto entrenamiento de los modelos.
\end{itemize}
