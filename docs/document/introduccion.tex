
\section{Introducción}
\subsection{Contexto}
De la mano del avance tecnológico tanto en materia de hardware como de software, en los últimos años ha sido posible explotar de forma más efectiva y eficiente una rama algo olvidada de la inteligencia artificial: las redes neuronales.

%# qué se viene logrando

Esta rama ha demostrado en múltiples ocasiones ser capaz de obtener resultados significativos en tareas de detección y localización de objetos, clasificación de escenas, segmentación de imágenes y detección de rostros (entre otras). 

% Esta rama ha demostrado en múltiples ocasiones ser capaz de obtener resultados significativos en tareas de detección [PAPER IMAGENET] y localización de objetos [PAPER OBJECT LOCALIZATION], clasificación de escenas [ALGUNOS DE LOS PAPERS UTILIZADOS], segmentación de imágenes [PAPERS IMAGE SEGMENTATION]y detección de rostros [PAPERS YOLO] (entre otras). 


%# en qué se aplica
Estas prácticas tienen una gran aplicación en la industria como la detección y localización de objetos como semáforos, transeúntes, automóviles y otros relacionados al tráfico para autos que se conducen por sí mismos, detección de rostros para controles de ingreso de personas a aeropuertos como también a empresas, detección de elementos aplicado a imágenes médicas (posibles tumores o malformaciones).


%# problemática a atacar

En este trabajo se hará frente a uno de los ejes inicialmente mencionados: la clasificación de escenas. Esta asignatura que resulta prácticamente trivial para una persona incluye un conjunto de actividades complejas por sí mismas: detección de objetos locales y su disposición dentro de la escena, entorno de fondo, distinción de características entre escenas parecidas, cultura de la que proviene la escena, calidad de reconocimiento de una escena y muchas más. Actualmente, mediante actividades de investigación, competiciones y necesidades del sector privado se han logrado significativos resultados en clasificación de escenas relacionadas a diferentes contextos: distinción entre lugares de una ciudad, clasificación de zonas de la misma a partir de imágenes satelitales, ambientes de una propiedad, escenas relacionadas al tráfico de una ciudad, etc.

%# escenas vs imágens
A priori, en una fugaz interpretación de la tarea a realizar, se la podría definir como la clasificación de imágenes tradicional, aunque no sea así. En esta actividad se puede destacar tanto detección de objetos como, en algunos casos, la definición de su contorno, sin importar el tamaño del mismo o su posicionamiento dentro de la imagen. La tarea de reconocer escenas acapara varias otras aristas, como son la disposición de los objetos en la imagen, los elementos que se encuentren en la misma, el ambiente en el que se encuentren, el fondo, entre otras. En una escena existen múltiples objetos en diferentes escalas, enfocados desde diferentes ángulos y disposiciones, mientras que en la clasificación de imágenes se suele tratar con un único objeto centrado. Éstas son, entre muchas otras, algunas de las principales diferencias entre clasificación de escenas e imágenes, dos tareas que pertenecen a un mismo tópico pero que no es posible solucionarlas totalmente utilizando el mismo enfoque para el problema.

%# cómo se venía haciendo antes VS ahora
Dado que hasta hace no muchos años la cantidad de imágenes a clasificar no alcanzaba el orden de magnitud que se tiene actualmente, queda claro que era posible de abordar la necesidad mediante tareas realizadas por individuos. Siendo que actualmente el flujo de información es mucho mayor, resulta que la automatización del etiquetado imágenes pasó a ser un requerimiento para determinados entornos como empresas de venta y/o alquiler de propiedades, intermediarios dentro de la misma actividad o sitios en los que se suben imágenes de este tipo y se las quiere mantener etiquetadas de forma inmediata.

%# qué se puede usar dentro de machine learning
Dentro de los posibles enfoques a utilizar descriptos en \cite{comparation_techniques} se encuentran las representaciones esparsas, máquinas de soporte vectorial, redes neuronales artificiales y redes neuronales convolucionales. Dentro de las redes neuronales se suelen usar diferentes arquitecturas en calidad de obtener los mejores resultados, dependiendo de en qué manera se estructure la información. Estas arquitecturas son las Redes Neuronales Convolucionales, las Redes Neuronales Recurrentes y el Aprendizaje por Transferencia.


\subsection{Motivación}\label{sec:motivacion}

La clasificación de imágenes relacionadas a propiedades inmuebles hace referencia a la capacidad de etiquetar automática y correctamente imágenes de escenas relacionadas a diferentes partes de los mismos de manera tal que luego sea posible consumir la información de cada sector de manera individual por cada bien. 
Una actividad que resulta altamente atractiva y beneficiosa cuando se trata con cientos o miles de imágenes de propiedades y se quiere explotar esta información para otro tipo de tareas. Dentro de los beneficios más destacables se pueden mencionar la automatización de tarea de etiquetado, las mejoras en sistemas que requieran este tipo de tareas, el ahorro de tiempo para clasificar imágenes de este dominio, entre otras.

En el contexto actual existen empresas que brindan una larga lista de servicios relacionados a las propiedades inmuebles y que la resolución de este problema les sería de gran ayuda tanto en las tareas del día a día como para explotar de mejor manera la información de los inmuebles que ya tienen almacenada internamente. Este tipo de empresas u organizaciones son las que se dedican a actividades como: la venta y alquiler de bienes propios, la intermediación entre residentes y dueños para alquileres temporales, la valuación y control del estado de las propiedades, entre otras.
Dentro de las posibles aplicaciones y ventajas que puede otorgar el adjudicarse con un modelo que se dedique a realizar esta activadad se destacan:
\begin{itemize}
	\item Clasificación automática de las escenas: el hecho de poder clasificar las escenas de cada propiedad de la que se tiene conocimiento permite no sólo un mejoramiento de calidad de información sino también la posibilidad de poder aprovecharla para ser explotada a posteriori.
	\item Validación de escenas requeridas: un claro ejemplo de mejora en el proceso de posteo de propiedades, ya sea para alquiler o para venta, es solicitar imágenes de los diferentes ambientes de la propiedad, validando mediante un modelo de este tipo que se provean imágenes de los ambientes que se consideren más importantes (por ejemplo, si se quiere postear un departamento en un sitio de alquileres, que se valide la existencia de imágenes de la cocina, el comedor, el dormitorio y el living, con el fin de mejorar la calidad de información que luego se brindará a quienes buscan alquilar). Sin un modelo encargado de esta tarea, esta mejora no es posible de obtenerse a gran escala.
	\item Extracción de características relevantes: a partir de un modelo de aprendizaje profundo, como se verá posteriormente, es posible extraer conocimiento para resolver otro tipo de problemas. Esta técnica se conoce Aprendizaje por Transferencia. 
	\item Recomendación de escenas: un posible enfoque luego de contar con las imágenes etiquetadas y puesto en producción un modelo que valide qué imagenes se postean de las propiedades, es posible realizar un análisis de qué relación existe entre las imágenes que la persona que alquila ve de la propiedad y la conversión del posteo (ya sea una venta, un contacto para visitarla o un alquiler temporal). De esta manera se podría brindar un mejor feedback a los propietarios para que aumenten sus probabilidades de convertir recomendando escenas faltantes o aquellas que mejores resultados podrían brindarle.
	\item Valoración automática o refinamiento de valoraciones de propiedades a partir de imágenes: en \cite{vision_based_real_estate_price_estimation} se demuestra que es posible mejorar los resultados obtenidos en tareas de valuación automática de inmuebles utilizando las imágenes de los ambientes de los mismos. 
	\item Otras aplicaciones que se pueden encontrar o se encontrarán en un futuro próximo en el mercado actual.
\end{itemize}

Dentro de la tarea previamente mencionada de valuación de inmuebles existen cientos de factores que componen el valor final de los mismos, tales como el terreno total ocupado y el construido, los años desde su construcción, la cantidad de habitaciones de cada tipo, los precios de los inmuebles circundantes y los precios de los inmuebles similares, el estado del inmueble, entre otros. Dada la complejidad de la cantidad de información a tener en cuenta y las diferentes formas en que se presenta (variables numéricas y categóricas, datos no estructurados como imágenes, etc), en conjunto con el alto número de inmuebles que se necesitan tasar en algunos casos se han implementado diferentes enfoques de valoración automática.

En \cite{vision_based_real_estate_price_estimation} se demuestra que a partir de la información que brindan las imágenes de los inmuebles y mediante aprendizaje profundo es posible mejorar los resultados de los enfoques que no cuentan con esta información y dedican su esfuerzo a realizar las tasaciones sólo con la información tabular y estructurada. 
Teniendo en cuenta este último punto, queda claro que poder absorber la mayor cantidad de información sobre estas imágenes resulta una tarea de alta significancia. Uno de los enfoques utilizados para hacerlo es basar la estimación del precio únicamente a partir de las imágenes que se tienen de la propiedad, y luego agregar los datos estructurados para mejorar los resultados.

Por otro lado, en actividades como el control o chequeo de imágenes que se suben al hacer publicaciones de propiedades o las sugerencias de escenas a subir dentro de las mismas, reconocer de qué habitación o vista se trata en cada caso resulta la única manera de brindar esta información al usuario.

Si de la agrupación de imágenes similares se trata, entonces luego de obtener la relación entre precio con el que se venden las propiedades y las características de las diferentes partes de las mismas sería posible conocer qué particularidades tienen aquellas propiedades que logran mayor margen de ganancia al momento de venderlas, qué se puede hacer con ciertas partes del inmueble para que su valor suba en relación a las cualidades de sus diferentes sectores y otras tareas relacionadas que brindarían un alto valor al momento de postular a la venta una propiedad.

Por último, es importante destacar la posibilidad de extraer el conocimiento adquirido por un modelo encargado de hacer estas tareas para hacer frente o refinar resultados de otros modelos, que no necesariamiente estén fuertemente relacionados con la clasificación de escenas de propiedades inmuebles.

Como se mostrará a partir de la revisión de antecedentes, la utilización de aprendizaje profundo para clasificar escenas es una opción altamente viable por los resultados que se obtienen. De igual manera, es cierto que aún existe un margen de error en estos modelos, que por un lado puede provenir por los datos que se utilizan para entrenarse y por otro por la gran cantidad de cuestiones a tener en cuenta en relación a las configuraciones de la red a utilizar (arquitectura, funciones de pérdida, optimizadores, métrica tener en cuenta, etc), lo que da lugar a continuar investigando con el fin de seguir refinando los modelos y eventualmente disminuir el margen de error en los resultados. Este trabajo hará foco en esta problemática, a través de la investigación de diferentes métodos de aprendizaje profundo que actualmente alcanzan el estado del arte, utilizando métricas que ya se utilicen en este tópico y conjuntos de datos para que tanto la diversidad como la densidad de las escenas sea la suficiente para brindar conclusiones correctas con respecto a esta tarea.


\subsection{Estructura del documento}
En este trabajo se investigarán diferentes métodos de clasificación de escenas aplicado a propiedades inmuebles. Con el fin de alcanzar el estado del arte en esta tarea y eventualmente intentar mejorar estos resultados, la composición del trabajo será la siguiente: en el capítulo segundo se realizará una revisión de antecedentes en materia de utilización de técnicas de aprendizaje profundo para clasificar escenas. En el capítulo tercero se definirá el marco teórico y otros conceptos necesarios para entender el funcionamiento de este tipo de técnicas. En el capítulo cuarto se enunciarán las limitaciones e hipótesis del proyecto, también se realizarán los experimentos para contrastar estas hipótesis. En el capítulo quinto se realizará un análisis de los resultados obtenidos por aquellos modelos que mejor funcionaron. En los capítulos sexto y séptimo de definirán los experimentos a realizar y sus resultados, respectivamente. Para finalizar, en el capítulo séptimo, se declararán las conclusiones del trabajo y posibles lineamientos futuros.