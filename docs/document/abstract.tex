\begin{abstract}

	En este trabajo se investigó sobre la aplicación de técnicas de Aprendizaje Profundo en el campo de la Clasificación de Escenas de Propiedades Inmuebles. Se revisaron trabajos previos sobre clasificación de escenas del mismo tipo y de otros con el fin de conocer qué enfoques podrían funcionar mejor. Se plantearon y comprobaron hipótesis que incluyen el entrenamiento de Redes Neuronales Convolucionales desde cero y el Aprendizaje Mediante Transferencia de Redes Neuronales Preentrenadas. A través de la validación de las hipótesis se obtuvieron modelos capaces de predecir determinados conjuntos de escenas de propiedades inmuebles. Se analizaron las predicciones de los dos mejores modelos obtenidos, haciendo énfasis en el error que cometen a partir de las probabilidades que entregan. Gracias a esta investigación se llegó a la conclusión de que es posible entrenar modelos de Aprendizaje Profundo para la clasificación de escenas pero que la capacidad de predicción estará sujeta a la calidad de las imágenes con las que se hayan entrenado.
	
	El presente documento, el código utilizado y los modelos entrenados quedan disponibles para uso bajo la Licencia MIT, en el \href{https://github.com/ifranco14/tf\_real\_estate\_images\_classification}{repositorio del trabajo}.
	
	{\bf Keywords:} Scene Classification, Deep Learning, Real Estate, Convolutional Neural Network, Transfer Learning.
	
	{\bf Palabras clave:} Clasificación de Escenas, Aprendizaje Profundo, Propiedades Inmuebles, Redes Neuronales Convolucionales, Aprendizaje por Transferencia.
\end{abstract}


